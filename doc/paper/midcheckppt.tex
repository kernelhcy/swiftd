\documentclass[dvipdfm]{beamer}

\usepackage{hcystyle}

%%%%%%%文档正文
\begin{document}

%标题作者等信息。
%这些信息将自动生成ppt的第一页
\title[中期答辩]{\LARGE{毕业设计中期检查答辩\\}}
%\subtitle{基于动态链接技术的{\cp web}服务器动态扩展功能\\接口的设计与实现}
\author[黄丛宇]{黄丛宇\\06161032\\指导老师:马瑞芳}
\institute[西安交通大学 软件学院]{西安交通大学 软件学院 软件62班}
\date{\texttt{\today}}

\begin{frame}	
	\titlepage
\end{frame}

\begin{frame}
	\frametitle{目录}
	\tableofcontents
\end{frame}

\section{课题名称}
%第二页
\begin{frame}
	\frametitle{课题名称}
	
	\begin{center}
	{\Large
			基于动态链接技术的{\cp web}服务器动态扩展功能\\
				接口的设计与实现
	}
	\end{center}
\end{frame}

\section{背景和意义}
\begin{frame}
	\frametitle{背景和意义}
	\begin{itemize}
		\item {\cp Web}服务器:
			\begin{itemize}
				\item[-] 互联网的核心组成部分,支撑整个互联网应用服务。
				\item[-] 适应互联网应用的不断更新变化。
				\item[-] 必须保证7*24小时的运行。
			\end{itemize}
		\pause
		\item {\cp Web}服务器现状:
			\begin{itemize}
				\item[-] 大部分都不支持功能的动态增加。
				\item[-] 必须重启或重新编译。
			\end{itemize}
	\end{itemize}
\end{frame}

\begin{frame}
	\frametitle{背景和意义}
	\begin{itemize}
		\item 针对以上问题,本课题将基于动态链接库技术,使服务器在运行期间,可以动态的获知模块的增加并加载模块。
		\pause
		\item 本系统实现了服务器的基本功能,重点实现模块动态加载特性。
	\end{itemize}
\end{frame}

\section{课题主要任务}
\begin{frame}
	\frametitle{课题主要任务}
	\begin{enumerate}
		\item 学习动态链接库和{\cp web}服务器设计实现的相关知识和技术。
		\begin{itemize}
			\item[-] {\cp Nonblocking IO, IO multiplexing},线程池,动态链接技术等。
		\end{itemize}
		\pause
		\item 技术可行性论证,需求分析,构造系统结构。
		\begin{itemize}
			\item[-] 分析{\cp HTTP}协议,设计接口,构造服务器的整体框架等。
		\end{itemize}
		\pause
		\item 搭建开发环境,设计和编码实现。
		\begin{itemize}
			\item[-] 完成服务器的编码。实现预期的功能。
		\end{itemize}
		\pause
		\item 撰写毕业设计论文
		\pause
		\item 查找外文文献并翻译。
	\end{enumerate}
\end{frame}

\section{已完成任务}
\begin{frame}
	\frametitle{已完成任务}
	\begin{itemize}
		\item 完成动态链接库技术和{\cp Web}服务器设计相关资料的搜集和学习。
		\pause
		\begin{itemize}
			\item[-] 掌握了动态链接库的基本使用。
			\item[-] 掌握了{\cp Nonblocking IO}和{\cp IO Multiplexing}的使用和注意事项。
			\item[-] 学习有关线程池的资料并编写了一个简单的线程池,以备后期使用。
		\end{itemize}
	\end{itemize}
\end{frame}

\begin{frame}
	\frametitle{已完成任务}
	\begin{itemize}
		\item 完成技术可行性论证,需求分析,构造系统结构。
		\pause
		\begin{itemize}
			\item[-] 研究学习了{\cp HTTP/1.1}协议({\cp RFC2616})。理解掌握了{\cp HTTP}协议的基本内容和处理过程。
			\item[-] 根据{\cp HTTP}协议的内容设计插件的接口。
			\item[-] 完成服务器整体架构的设计。
			\pause
			\setbeamertemplate{items}[circle]
			\begin{itemize}
					\item 对于连接的处理采用状态机。
					\item 每个{\cp IO}事件调用一个线程进行处理。
					\item 采用{\cp Inotify}检测插件的增加和删除。
			\end{itemize}
			
		\end{itemize}
	\end{itemize}
\end{frame}

\begin{frame}
	\frametitle{已完成任务}
	%
	\begin{itemize}
		\item 完成开发环境的搭建。完成服务器设计并编码实现。
		\begin{itemize}
			\item[-] 开发环境:{\cp Linux(Debian5.0)+gcc+vim(gedit)}
			\item[-] 完成服务器的编码工作。
			\pause
			\setbeamertemplate{items}[circle]
			\begin{itemize}
					\item 实现了{\cp HTTP/1.1}的{\cp GET, POST}和{\cp HEAD}方法。
					\item 实现了{\cp HTTP/1.1}中的{\cp Persistent Connections}和{\cp Pipelining}。
					\item 实现了{\cp HTTP/1.0}中的{\cp Nonpersistent Connections}。
					\item 可以动态的加载和删除插件。
					\item 实现了一个验证性质的插件:{\cp dir\_index}。
			\end{itemize}
		\end{itemize}
	\end{itemize}
\end{frame}

\begin{frame}
	\begin{center}
		\frametitle{完成情况}
		\pause
		\LARGE{目前已经完成课题任务的}\\
		{\Huge  80\% }
		\\
	\end{center}
\end{frame}

\section{存在的问题}
\begin{frame}
	\frametitle{存在的问题}
	\begin{enumerate}
		\item 对{\cp HTTP}协议的理解不够透彻。
		\begin{itemize}
			\item[-] 缓存
			\item[-] 条件获取
		\end{itemize}
		\pause
		\item 代码存在{\cp BUG}。
		\begin{itemize}
			\item[-] 内存泄漏
			\item[-] 线程并发出现数据混乱
		\end{itemize}
	\end{enumerate}
\end{frame}

\section{后期工作计划}
\begin{frame}
	\frametitle{后期工作计划}
	\begin{enumerate}
		\item 继续深入研究学习{\cp HTTP}协议。
		\item 调成程序,修改{\cp BUG}。
		\item 编写毕业设计论文。
		\item 查找翻译外文资料。
		\item 准备毕业设计答辩文档。
	\end{enumerate}
\end{frame}

\begin{frame}
	\frametitle{结束}
	\begin{center}
	{\Huge
		\textsl{{\cp That's all!}}
	}
	\end{center}
\end{frame}

\end{document}
