\documentclass[12pt]{beamer}
\usepackage{CJKutf8} 	%支持汉字
\usetheme{Madrid}		%使用主题

%%%%%%%文档正文
\begin{document}
\begin{CJK}{UTF8}{gbsn}	%宋体

%标题作者等信息。
%这些信息将自动生成ppt的第一页
\title{毕业设计中期检查}
\author{黄丛宇}
\institute{西安交通大学 软件学院 软件62}
\date{\today}
\frame{\titlepage}

%第二页
\begin{frame}
	\frametitle{课题名称}
	\begin{center}
	{
	\Large
		\textbf{
			基于动态链接技术的web服务器动态扩展功能%
				接口的设计与实现
		}
	}
	\end{center}
\end{frame}

\begin{frame}
	\frametitle{背景和意义}
	\begin{itemize}
		\item Web服务器:
			\begin{itemize}
				\item[-] 互联网的核心组成部分,支撑整个互联网应用服务。
				\item[-] 适应互联网应用的不断更新变化。
				\item[-] 必须保证7*24小时的运行。
			\end{itemize}
		\pause
		\item Web服务器现状:
			\begin{itemize}
				\item[-] 大部分都不支持功能的动态增加。
				\item[-] 必须重启或重新编译。
			\end{itemize}
	\end{itemize}
\end{frame}

\begin{frame}
	\frametitle{背景和意义}
	\begin{itemize}
		\item 针对以上问题,本课题将基于动态链接库技术,使服务器在运行期间,可以动态的获知模块的增加并加载模块。
		\pause
		\item 本系统实现了服务器的基本功能,重点实现模块动态加载特性。
	\end{itemize}
\end{frame}

\begin{frame}
	\frametitle{课题主要任务}
	\begin{enumerate}
		\item 学习动态链接库和web服务器设计实现的相关知识和技术。
		\begin{itemize}
			\item[-] Nonblocking IO, IO multiplexing,线程池,动态链接技术等。
		\end{itemize}
		\pause
		\item 技术可行性论证,需求分析,构造系统结构。
		\begin{itemize}
			\item[-] 分析HTTP协议,设计接口,构造服务器的整体框架等。
		\end{itemize}
		\pause
		\item 搭建开发环境,设计和编码实现。
		\begin{itemize}
			\item[-] 完成服务器的编码。实现预期的功能。
		\end{itemize}
		\pause
		\item 撰写毕业设计论文
		\pause
		\item 查找外文文献并翻译。
	\end{enumerate}
\end{frame}

\begin{frame}
	\frametitle{已完成任务}
	\begin{itemize}
		\item 完成动态链接库技术和Web服务器设计相关资料的搜集和学习。
		\pause
		\begin{itemize}
			\item[-] 掌握了动态链接库的基本使用。
			\item[-] 掌握了Nonblocking IO和IO Multiplexing的使用和注意事项。
			\item[-] 学习有关线程池的资料并编写了一个简单的线程池,以备后期使用。
		\end{itemize}
	\end{itemize}
\end{frame}

\begin{frame}
	\frametitle{已完成任务}
	\begin{itemize}
		\item 完成技术可行性论证,需求分析,构造系统结构。
		\pause
		\begin{itemize}
			\item[-] 研究学习了HTTP/1.1协议(RFC2616)。理解掌握了HTTP协议的基本内容和处理过程。
			\item[-] 根据HTTP协议的内容设计插件的接口。
			\item[-] 完成服务器整体架构的设计。
			\pause
			\setbeamertemplate{items}[circle]
			\begin{itemize}
				\item 对于连接的处理采用状态机。
				\item 每个IO事件调用一个线程进行处理。
				\item 采用Inotify检测插件的增加和删除。
			\end{itemize}
		\end{itemize}
	\end{itemize}
\end{frame}

\begin{frame}
	\frametitle{已完成任务}
	%
	\begin{itemize}
		\item 完成开发环境的搭建。完成服务器设计并编码实现。
		\begin{itemize}
			\item[-] 开发环境:Linux(Debian5.0)+gcc+vim(gedit)
			\item[-] 完成服务器的编码工作。
			\pause
			\setbeamertemplate{items}[circle]
			\begin{itemize}
				\item 实现了HTTP/1.1的GET,POST和HEAD方法。
				\item 实现了HTTP/1.1中的Persistent Connections和Pipelining。
				\item 实现了HTTP/1.0中的Nonpersistent Connections。
				\item 可以动态的加载和删除插件。
				\item 实现了一个验证性质的插件:dir\_index。
			\end{itemize}
		\end{itemize}
	\end{itemize}
\end{frame}

\begin{frame}
	\begin{center}
		\frametitle{完成情况}
		\pause
		\LARGE{目前已经完成课题任务的}\\
		{\Huge \textbf{80\%} }
		\\
	\end{center}
\end{frame}

\begin{frame}
	\frametitle{存在的问题}
	\begin{enumerate}
		\item 对HTTP协议的理解不够透彻。
		\begin{itemize}
			\item[-] 缓存
			\item[-] 条件获取
		\end{itemize}
		\pause
		\item 代码存在BUG。
		\begin{itemize}
			\item[-] 内存泄漏
			\item[-] 线程并发出现数据混乱
		\end{itemize}
	\end{enumerate}
\end{frame}

\begin{frame}
	\frametitle{后期工作计划}
	\begin{enumerate}
		\item 继续深入研究学习HTTP协议。
		\item 调成程序,修改BUG。
		\item 编写毕业设计论文。
		\item 查找翻译外文资料。
		\item 准备毕业设计答辩文档。
	\end{enumerate}
\end{frame}

\begin{frame}
	\frametitle{结束}
	\begin{center}
	{\Huge
		\textit{That's all!}
	}
	\end{center}
\end{frame}

\end{CJK}
\end{document}
