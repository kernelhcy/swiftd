\documentclass[12pt, a4paper]{article}

\usepackage{fontspec, xunicode, xltxtra}
\usepackage{hyperref}

%中英混排
\usepackage{xeCJK}
\setmainfont{Courier 10 Pitch} 				%默认字体,默认英文字体。
\setCJKmainfont{WenQuanYi Zen Hei} 			%中文默认字体
\CJKsetecglue{}

\usepackage{enumerate}

\begin{document}

\begin{center}
{\Huge 2010年中期答辩总结}\\
{\Large ---软件62班---黄丛宇}
\end{center}

\section{评委问题:}
\begin{enumerate}[1)]
	\item 本系统是属于应用级还是系统级。
	\item 系统如何数据库等进行交互。
	\item 如何对系统进行验证。如,怎样验证动态加载功能。
\end{enumerate}

\section{问题解决措施:}
\begin{enumerate}[1)]
	\item 本系统输入系统级。
	\item 系统不涉及数据库。
	\item 对于服务器的基本功能,系统将会包含一个测试性的网页,通过浏览器访问网页来验证服务器的基本功能。对于服务器的负载能力,主要通过使用测试程序在短时间内创建大量请求进行测试。对于动态加载功能的验证,程序中包含一个验证性的模块---dir\_index。这个模块可以使得请求仅仅定位到目录,服务器自动搜索目录中的index.html等文件。
\end{enumerate}

\section{中期经验和教训:}
\begin{enumerate}[1)]
	\item PPT做的不够美观。尤其是缺少图表。
	\item 对于评委的问题准备的不够充分。
	
\end{enumerate}

\section{后期计划和安排:}
\begin{enumerate}[1)]
	\item 继续修改系统中的BUG。
	\item 学习使用latex制作PPT和毕业设计论文。
	\item 翻译外文资料。
\end{enumerate}

\end{document}
