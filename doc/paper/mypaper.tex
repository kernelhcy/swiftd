\documentclass[18pt, twoside, a4paper, dvipdfm]{book}

\usepackage{hcypaperstyle}

\begin{document}

\title{{\Huge 本科毕业设计论文\\}基于动态链接技术的web服务器动态扩展功能接口的设计与实现}
\author{黄丛宇\\06161032\\指导老师:马瑞芳}
\date{\today}

\maketitle

\lfour

\zhabstract 

Web服务器作为互联网的核心组成部分,在支撑整个互联网服务中,起到了至关重要的作用。由于互联网中应用的不断更新,要求web服务器也能适应这种更新变化。同时,web服务器必须保证7*24小时的运行。在目前的正在使用的web服务器中,大部分都不支持功能的动态增加,也就是在不重启服务器的情况下,动态的为服务器增加功能。针对这一问题,本课题基于动态链接库技术,使服务器在运行期间,可以动态的获知模块的增加并加载模块。本系统实现了服务器的基本功能,重点实现模块动态加载特性,并对实现进行测试。 

本论文的主要内容是研究实现基于动态链接技术的Web服务器动态加载插件的接口。首先,文中介绍了本课题的背景和意义,以及作者的主要任务和工作。接着概述与本课题相关的HTTP协议(RFC2616)的内容,简单介绍动态链接技术及其使用,还有介绍Web服务器的一些主流设计和实现方法。接下来,通过对HTTP协议的分析,设计插件的接口,分析动态链接技术,设计动态加载插件的过程。然后,采用非阻塞I/O+线程池+I/O多路复用技术,设计服务器的架构。然后,根据以上的设计,在Linux系统下,实现服务器的原型,并在其上实现插件接口和动态加载功能。

最后,通过测试和观察运行结果,分析服务器是否可以正确处理HTTP请求,包括HTTP/1.1的持久连接和HTTP/1.0的非持久连接,并返回客户端所请求的资源。同时,验证插件接口的设计能否满足扩展功能的需求,能否正确的动态加载插件。本系统可以处理一般规模的请求数量。

{\zhkeywords Web服务器;插件;Linux;C;动态加载}

\enabstract
I will translate it later...

{\enkeywords Web Server;Plugin;Linux;C;Dynamic load}

\newpage
\tableofcontents

\chapter{绪论}

\section{研究背景和意义}
\subsection{Web服务器的背景}
	Web服务器也称为WWW(Word Wide Web)服务器,主要功能是提供网上信息浏览服务。Web服务器通过解析处理HTTP协议完成客户端的请求。当Web服务器接收到一个HTTP请求时,通过解析请求的内容,经过一些列的处理(如,查找客户端请求的资源),生成一个HTTP响应并发送给客户端。这个HTTP响应包含客户端所请求的数据(例如送回一个HTML页面)或者提示信息等。在处理一个请求,Web服务器可以返回给客户端一个静态页面或图片,进行页面跳转,或者把动态响应委托给一些其它的程序处理(例如CGI脚本,JSP脚本,ASP脚本,或者一些其它的服务器端技术)。这些处理委托请求的服务器端程序通常产生一个HTML的响应,并通过web服务器发送给客户端。
	
	目前,主流的web服务器包括Apache,Lighttpd和Nginx。
	\begin{itemize}
		\item Apache是世界排名第一的web服务器,是Apache软件基金会的一个开放源码的web服务器。Apache可以运行在几乎所有的计算机平台上,如,Linux,Windows,Unix等。Apache快速、可靠并且可通过简单的API扩充。Apache web服务器具有如下特点:
		\begin{enumerate}
			\item 支持最新的HTTP/1.1通信协议
  			\item 拥有简单而强有力的基于文件的配置过程
  			\item 支持通用网关接口和FastCGI
  			\item 支持基于IP和基于域名的虚拟主机
  			\item 支持多种方式的HTTP认证
  			\item 集成Perl处理模块
  			\item 支持实时监视服务器状态和定制服务器日志
  			\item 支持服务器端包含指令(SSI)
  			\item 支持安全Socket层(SSL)
  			\item 提供用户会话过程的跟踪
  			\item 通过第三方模块可以支持Java Servlets
		\end{enumerate}
		\item Lighttpd是一个德国人领导的开源软件,根本在于提供一个专门针对高性能网站,安全、快速、兼容性好并且灵活的web服务器。Lighttpd具有非常低的内存开销,cpu占用率低,较好的性能以及丰富的模块。Lighttpd主要针对Unix/Linux平台,通过Cygwin也可以运行在Windows平台上,具有较好的夸平台特性。
		\item Nignx
	\end{itemize}
\subsection{动态链接技术背景}

\section{课题目标及作者的主要工作}

\chapter{服务器设计和实现技术}

\section{HTTP协议}
\section{动态链接技术}
\section{非阻塞I/O与I/O多路复用}
\section{线程池}
\section{状态机}

\chapter{HTTP协议的分析与插件接口的设计}

\section{HTTP协议分析}
\subsection{HTTP协议的数据格式}
\subsection{HTTP协议的处理过程}

\section{插件接口的设计}
\subsection{插件接口的定义}
\subsection{插件动态加载过程}
\subsection{插件调用的过程}

\chapter{Web服务器的设计}
\section{I/O设计}
\section{链接处理}

\chapter{插件接口的实现}
\section{接口定义实现}
\section{动态加载过程的实现}
\section{插件调用的实现}

\chapter{服务器的实现}
\section{I/O实现}
\section{线程池的实现}
\section{连接处理状态机的实现}

\chapter{结束语}
\section{总结}
\section{展望}

\chapter*{致谢}
\chapter*{参考文献}

\end{document}
